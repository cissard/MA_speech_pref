\documentclass[man]{apa6}
\usepackage{lmodern}
\usepackage{amssymb,amsmath}
\usepackage{ifxetex,ifluatex}
\usepackage{fixltx2e} % provides \textsubscript
\ifnum 0\ifxetex 1\fi\ifluatex 1\fi=0 % if pdftex
  \usepackage[T1]{fontenc}
  \usepackage[utf8]{inputenc}
\else % if luatex or xelatex
  \ifxetex
    \usepackage{mathspec}
  \else
    \usepackage{fontspec}
  \fi
  \defaultfontfeatures{Ligatures=TeX,Scale=MatchLowercase}
\fi
% use upquote if available, for straight quotes in verbatim environments
\IfFileExists{upquote.sty}{\usepackage{upquote}}{}
% use microtype if available
\IfFileExists{microtype.sty}{%
\usepackage{microtype}
\UseMicrotypeSet[protrusion]{basicmath} % disable protrusion for tt fonts
}{}
\usepackage{hyperref}
\hypersetup{unicode=true,
            pdftitle={Infants prefer to listen to speech: A meta-analysis.},
            pdfauthor={Cécile Issard, Sho Tsuji, \& Alejandrina Cristia},
            pdfkeywords={Meta-analysis, infants, speech preference, auditory development, natural
sounds},
            pdfborder={0 0 0},
            breaklinks=true}
\urlstyle{same}  % don't use monospace font for urls
\usepackage{graphicx,grffile}
\makeatletter
\def\maxwidth{\ifdim\Gin@nat@width>\linewidth\linewidth\else\Gin@nat@width\fi}
\def\maxheight{\ifdim\Gin@nat@height>\textheight\textheight\else\Gin@nat@height\fi}
\makeatother
% Scale images if necessary, so that they will not overflow the page
% margins by default, and it is still possible to overwrite the defaults
% using explicit options in \includegraphics[width, height, ...]{}
\setkeys{Gin}{width=\maxwidth,height=\maxheight,keepaspectratio}
\IfFileExists{parskip.sty}{%
\usepackage{parskip}
}{% else
\setlength{\parindent}{0pt}
\setlength{\parskip}{6pt plus 2pt minus 1pt}
}
\setlength{\emergencystretch}{3em}  % prevent overfull lines
\providecommand{\tightlist}{%
  \setlength{\itemsep}{0pt}\setlength{\parskip}{0pt}}
\setcounter{secnumdepth}{0}
% Redefines (sub)paragraphs to behave more like sections
\ifx\paragraph\undefined\else
\let\oldparagraph\paragraph
\renewcommand{\paragraph}[1]{\oldparagraph{#1}\mbox{}}
\fi
\ifx\subparagraph\undefined\else
\let\oldsubparagraph\subparagraph
\renewcommand{\subparagraph}[1]{\oldsubparagraph{#1}\mbox{}}
\fi

%%% Use protect on footnotes to avoid problems with footnotes in titles
\let\rmarkdownfootnote\footnote%
\def\footnote{\protect\rmarkdownfootnote}


  \title{Infants prefer to listen to speech: A meta-analysis.}
    \author{Cécile Issard\textsuperscript{1}, Sho Tsuji\textsuperscript{2}, \&
Alejandrina Cristia\textsuperscript{1}}
    \date{}
  
\shorttitle{Preference for speech sounds in infancy}
\affiliation{
\vspace{0.5cm}
\textsuperscript{1} Laboratoire de Sciences Cognitives et Psycholinguistique, Ecole Normale Supérieure, Département d'Études Cognitives\\\textsuperscript{2} International Research Center for Neurointelligence, The University of Tokyo}
\keywords{Meta-analysis, infants, speech preference, auditory development, natural sounds\newline\indent Word count: X}
\usepackage{csquotes}
\usepackage{upgreek}
\captionsetup{font=singlespacing,justification=justified}

\usepackage{longtable}
\usepackage{lscape}
\usepackage{multirow}
\usepackage{tabularx}
\usepackage[flushleft]{threeparttable}
\usepackage{threeparttablex}

\newenvironment{lltable}{\begin{landscape}\begin{center}\begin{ThreePartTable}}{\end{ThreePartTable}\end{center}\end{landscape}}

\makeatletter
\newcommand\LastLTentrywidth{1em}
\newlength\longtablewidth
\setlength{\longtablewidth}{1in}
\newcommand{\getlongtablewidth}{\begingroup \ifcsname LT@\roman{LT@tables}\endcsname \global\longtablewidth=0pt \renewcommand{\LT@entry}[2]{\global\advance\longtablewidth by ##2\relax\gdef\LastLTentrywidth{##2}}\@nameuse{LT@\roman{LT@tables}} \fi \endgroup}


\DeclareDelayedFloatFlavor{ThreePartTable}{table}
\DeclareDelayedFloatFlavor{lltable}{table}
\DeclareDelayedFloatFlavor*{longtable}{table}
\makeatletter
\renewcommand{\efloat@iwrite}[1]{\immediate\expandafter\protected@write\csname efloat@post#1\endcsname{}}
\makeatother
\usepackage{lineno}

\linenumbers

\authornote{

Correspondence concerning this article should be addressed to Cécile
Issard, Laboratoire de Sciences Cognitives et Psycholinguistique,
Département d'Études Cognitives, Ecole Normale Supérieure, 29 rue d'Ulm,
75005 Paris, France. E-mail:
\href{mailto:cecile.issard@gmail.com}{\nolinkurl{cecile.issard@gmail.com}}}

\abstract{

}

\begin{document}
\maketitle

\subsection{Aknowledgement}\label{aknowledgement}

This work was supported by an Agence Nationale de la Recherche grant to
A.C. (ANR-17-CE28-0007 LangAge, ANR-16-DATA-0004 ACLEW, ANR-14-CE30-0003
MechELex, ANR-17-EURE-0017); and the J. S. McDonnell Foundation
Understanding Human Cognition Scholar Award to A.C.

\section{Introduction}\label{introduction}

Given the importance of speech for human vocal communication, it is
conceivable that infants are born equipped with a preference for speech,
which becomes stronger with age and exposure. Here, we synthesize
empirical data on infants' preferences for speech over non-speech sounds
to assess the explanatory role of three factors: stimulus naturalness,
vocal quality, and familiarity.

{[}insert Figure 1 here{]}

\subsection{Potential dimensions underlying preference
patterns}\label{potential-dimensions-underlying-preference-patterns}

There are three key conceptual explanations for infants preference for
speech over competitor sounds: Preference for (a) natural over
artificial sounds; (b) familiar over unfamiliar sounds; and (c) vocal
over non-vocal sounds (Figure 1). These explanations are mutually
compatible, and one or more may be true.

\subsubsection{Natural versus artificial
sounds}\label{natural-versus-artificial-sounds}

Natural sounds are those produced by biological systems, including vocal
tracts but also the sound of walking and heart rate. In many cases
natural and artificial sounds differ in their acoustic characteristics.
For example, backward speech has unnatural formant transitions and
seemingly abrupt closures compared to naturally produced sounds. Natural
sounds are processed more accurately by the auditory system, from the
cochlea (Lewicki, 2006) to the auditory cortex (e.g.~Mizrahi et al.,
2014). This predicts a preference for speech over artificial competitors
that is present from birth, consistent with existing literature (e.g.,
Vouloumanos \& Werker, 2007).

\subsubsection{Familiarity}\label{familiarity}

Perhaps infants prefer speech to other sounds since it is a frequent
sound. Newborns prefer their native speech to prosodically distinct
foreign speech (e.g., Mehler et al., 1988), which supports a preference
for sound patterns heard frequently in the womb. There are no behavioral
results directly testing the prediction that infants show stronger
preferences for speech over a foil when tested with more familiar speech
stimuli (for instance, spoken in their native, as compared to a foreign
language), but results from neuroimaging studies provide indirect
evidence for this view. For instance, newborns' brain activation was
different for forward than backward speech when the native language was
used as the speech stimuli, but not when a foreign language was used
(May et al., 2018).

\subsubsection{Vocal versus non-vocal
sounds}\label{vocal-versus-non-vocal-sounds}

Many results summarized above may be accommodated by a third hypothesis,
postulating a preference for vocal over non-vocal sounds. Vocal sounds
are those made with a mouth, and thus typically a subset of natural
sounds. Newborns made more head-turns to speech than to heartbeat
(Ecklund-Flores \& Turkewitz, 1996), arguably both equally natural and
familiar to them, but they listened equivalently to speech and monkey
calls, despite the greater familiarity of the former (Vouloumanos \&
Werker, 2010).

\subsubsection{Changes as a function of
development}\label{changes-as-a-function-of-development}

Development may affect the preference for speech in various ways.
Whereas newborns do not prefer speech over monkey calls,
three-month-olds do (Vouloumanos et al., 2010). This suggests that as
they age, infants might develop an increasingly narrow definition of the
stimulus they prefer. In this case naturalness, familiarity, and vocal
quality effects should change as a function of age: Very close stimuli
(e.g., speech versus another natural sound) initially leads to a weak
preference, but, as infants age, this preference may be as strong as
that found for very different stimuli (e.g., speech versus an artificial
sound). Many articles discuss potential changes in the pattern of
preference as a function of age (e.g.~Ferry et al., 2013; Shultz \&
Vouloumanos, 2010; Shultz et al., 2014). To our knowledge, only two
papers from the same laboratory include multiple age groups tested with
the exact same stimulus categories and procedure (Vouloumanos et al.,
2010; Vouloumanos \& Werker, 2004). In fact, statements about
age-related changes are often done using the demonstrably problematic
method of concluding that there is an interaction without actually
testing for it statistically (Gelman \& Stern, 2012). It is therefore
important to directly test these statements.

\subsection{A meta-analytic approach}\label{a-meta-analytic-approach}

In sum, previous work on infants' preferences is broadly compatible with
preference for natural over artificial, vocal over non-vocal, and
familiar over unfamiliar sounds, potentially interacting with infants'
age. In this paper, we seek to directly test these interpretations of
the literature by employing a meta-analytic approach. Meta-analyses
involve combining studies that may vary in their methodology. One
limitation is therefore that one cannot isolate specific variables as
well as in direct experimentation. Therefore, meta-analyses may miss
subtle effects. Nonetheless, they have several useful features. They can
reveal small effects not obvious in individual studies by combining them
to obtain larger samples. Additionally, by integrating data across
different laboratories, they provide evidence for the generalizability
of effects across labs. Finally, meta-analyses offer tools to detect
publication bias in the literature.

Specifically for the present case, a meta-analysis allows to
statistically test different explanations. We can test the effect of
factors that are not part of the original design, by redescribing the
stimuli used as a function of those factors. For instance, a study
measuring preference for native speech over native backward speech
provides data on a natural versus artificial, as well as a vocal versus
non-vocal contrast. We can also draw a developmental timeline across the
age range covered by the literature.

Meta-analyses can even provide theoretical and empirical insights that
contradict qualitative reviews. For example, it has been proposed that
infants' preference for novel or familiar items related to infants' age
such that, all things equal, younger infants showed familiarity
preferences whereas older infants exhibited novelty preferences (Hunter
\& Ames, 1988). However, Bergmann and Cristia (2016?) found stable
familiarity preferences for word segmentation in natural speech across
the first two years; and Black and Bergmann (2016) found a stable
novelty effect for artificial grammars implemented in synthesized speech
whereas those implemented in natural speech led to stable familiarity
preferences. Meta-analyses are therefore important to statistically and
systematically test the theoretical predictions proposed in qualitative
reviews.

Given the scarcity of direct evidence on the potential explanations laid
out above (naturalness, vocal quality, and familiarity, as a function of
age), we conducted a meta-analysis to test whether infants' preference
for speech sounds over other types of sounds is stable in newborns, and
how it develops over the first year of life. Assuming all three factors
are true, and further assuming that the definition of the preferred
stimulus narrows with age, we predicted that infants will show (see
Figure 2): 1. a greater preference for speech over natural sounds as a
function of age, but a stable preference for speech over artificial
sounds that is stable over development, 2. but an increasing preference
for speech over natural sounds; 3. a greater preference for speech over
other vocal sounds as a function of age, but a stable preference for
speech over non-vocal sounds over development; 4. a greater preference
for native speech over non-speech as a function of age, but a smaller
preference for foreign speech over non-speech with age .

\section{Methods}\label{methods}

\subsection{Literature search}\label{literature-search}

We composed the initial list of studies with suggestions by experts
(authors of this work); one google scholar searches ( (\enquote{speech
preference} OR \enquote{own-species vocalization}) AND infant -
\enquote{infant-directed}), the same search in PubMed and PsycInfo (last
searches on 24/09/2019); and a google alert. We also inspected the
reference lists of all included papers.

\subsection{Inclusion criteria}\label{inclusion-criteria}

After a first screening based on titles and abstracts using more liberal
inclusion criteria, we decided on inclusion based on full paper reading.
We included studies that tested human infants from birth to 1 year of
age, and contrasted speech sounds with any other type of sound,
measuring behavioral responses to the sounds (e.g., looking times). We
excluded studies that only contrasted foreign against native language,
did not present natural speech sounds at all, presented speech in the
mother's voice, or intentionally mixed speech with other vocal sounds
within the same sound condition. We also excluded neuroimaging studies
to avoid mixing results from different brain regions with different
response profiles. We included published (i.e., journal articles) as
well as unpublished works (i.e., doctoral dissertations as long as
sufficient information was provided).

A PRISMA flow chart summarizes the literature review and selection
process (Figure 3). We documented all the studies that we inspected in a
decision spreadsheet (available in the online supplementary materials;
Anonymized, 2019).

{[}Insert Figure 3 here{]}

\subsection{Coding}\label{coding}

The critical variables for our purpose are infant age, methodological
variables (testing method: central fixation, high amplitude sucking,
head-turn preference procedure, high amplitude sucking/passive
listening), and key stimuli characteristics. Specifically, we coded the
language in which the speech sounds were recorded (native or foreign),
and whether the sound opposed to speech was natural or not, vocal or
not. This competitor was coded as natural if it was produced by a
biological organism without any further acoustic manipulation. If the
authors applied acoustic manipulations it was coded as artificial. This
sound was considered as vocal if it was produced by an animal vocal
tract, either original or modified.

Data were coded by the first author. In addition, 20\% of the papers
were randomly selected to be coded by the second author independently,
with disagreements resolved by discussion. There were 10 disagreements
out of a total of 260 fields filled in indicative of the coders not
following the codebook, which led to a revision of all data in four
variables.

We coded all the statistical information reported in the included
papers. If reported, we coded the mean score and the standard deviation
for speech, and the other sound separately. When infant-level data was
provided, we recomputed the respective mean scores and standard
deviations based on the reported individual scores. If reported, we also
coded the t-statistic between the two sound conditions, or an
F-statistic provided this was a two-way comparison. If effect sizes were
directly reported as a Cohen's d or a Hedges' g, we also coded this.

The PRISMA checklist, data, and code can be found on the online
supplementary materials
(\href{https://osf.io/4stz9/?view_only=d0696591ebf34bfc8430f848cd945ca8}{Anonymized},
2019).

\subsection{Effect sizes}\label{effect-sizes}

Once the data were coded, we extracted effect sizes, along with their
respective variance. Effect sizes were standardized differences (Cohen's
d) between response to speech and to the other sound. If they were not
directly reported in the papers, we computed them using the respective
means and SDs (({\textbf{???}}) \& Wilson, 2001), or a t- or F-statistic
(({\textbf{???}}) et al., 1996). As our effect sizes came from
within-subject comparisons (e.g.~looking time of the same infant during
speech and during monkey calls), we needed to take into account the
correlation between the two measurements in effect sizes and effect size
variances computations. We computed this correlation based on the
t-statistic, the respective means and SDs (({\textbf{???}}) \& Wilson,
2001) if they were all reported; or imputed this correlation randomly if
not. We finally calculated the variance of each effect size
(({\textbf{???}}) \& Wilson, 2001). Cohen's d were transformed to
Hedges' g by multiplying d by a correction for small sample sizes based
on the degree of freedom (Borenstein et al., 2011). This procedure led
to XX (some of them coming from the same infant group, hence not
mutually independent) effect sizes, XX from published, peer-reviewed
paper, and 1 from a thesis.

All analyses use the R {[}CITER{]} package Robumeta (({\textbf{???}}) et
al., 2010), which allows to fit meta-analytic regressions that take into
account the correlated structure of the data, when repeated measures are
obtained from the same infant groups within papers.

\section{Results}\label{results}

\subsection{Database description}\label{database-description}

We found a total of 28 papers reporting 47 (not mutually independent)
effect sizes, see Table 1. 28 papers have been submitted to or published
in peer-reviewed journals (citations). The remaining 1 paper was a
thesis (({\textbf{???}})).

Studies tended to have small sample sizes, with a median N of 16
children (Range = 52, M = 20.66, Total: 659). Infants ranged from 0 to
12 months (1.50 to 380.50 days), although the majority were under 9
months of age (71.42\% of the studies). Individual samples comprised 48
\% of female participants on average. Infants were native of 7 different
languages across the whole database (English, French, Japanese, Italian,
Russian, Yiddish, Hebrew). Studies were performed in 13 different
laboratories from 6 different countries (United States, Canada, Israel,
France, Japan, Italy). 3 experimental methods were used: 12 studies used
Central Fixation (CF) (({\textbf{???}}) \& Aslin,1994; ({\textbf{???}})
\& Werker (2004); ({\textbf{???}}) et al. (2010); ({\textbf{???}}) \&
Vouloumanos (2010); ({\textbf{???}}) \& Bundy (1981); ({\textbf{???}})
et al. (2009); ({\textbf{???}}) et al. (2019); ({\textbf{???}}) et al.
(2019); ({\textbf{???}}) \& Kishon-Rabin (2011); ({\textbf{???}}) \&
Vouloumanos (2013); ({\textbf{???}}) \& Curtin (2014); ({\textbf{???}}),
2018); 3 used High-Amplitude Sucking (HAS) (({\textbf{???}}) \& Werker,
2007; ({\textbf{???}}) et al., 2010; ({\textbf{???}}) \& DeCasper,
1987); and 1 used Head-turn Preference Procedure (HPP) (({\textbf{???}})
\& Turkewitz, 1996).

\subsection{Summary effect size}\label{summary-effect-size}

Integrating across all studies in a meta-analytic regression without any
moderator, we found a summary effect size g of 0.53 (SE = 0.09 CI =
{[}0.34 , 0.73{]}), corresponding to a medium effect size.

\subsection{Publication bias}\label{publication-bias}

We assessed the presence of a potential publication bias in the body of
literature by studying the relationship between standard errors of
effect sizes as a function of Hedges' g (see funnel plot in Figure 4). A
regression test on these data was significant (z = 6.40, p \textless{}
0.0001), as was the Kendall's tau rank correlation test for funnel plot
asymmetry (Kendall's tau = 0.5172, p \textless{} .0001), indicating a
publication bias in the literature. To further investigate this bias, we
symmetrized the funnel plot with the \enquote{trim and fill} method
(Duval, 2005). XX18 (SE = XX5.87) missing studies were needed on the
left side of the plot to symmetrize the funnel plot.

{[}insert Figure 4 here{]}

\subsection{Moderator analyses}\label{moderator-analyses}

We then tested if the preference found above could be explained by the
dimensions discussed in the literature. Following our hypothesis, we fit
a meta-analytic model with the following moderators:

\begin{itemize}
\tightlist
\item
  mean age of children;
\item
  familiarity with the language used (native or foreign);
\item
  naturalness of the contrastive sound (coded as yes if it was natural
  and no otherwise).
\item
  vocal quality of the contrastive sound (coded as yes if it was vocal
  and no otherwise).
\end{itemize}

There was only a significant effect of the experimental method used
(central fixation led to higher effect sizes than the other methods; see
Figure 5). None of the other moderators was significant (see Table 1).

Due to the relatively low number of effect sizes available in the
literature, we did not add interactions with age to the model to avoid
overfitting. Inspection of results in Figure 5 show that the confidence
intervals of all conditions overlap almost exactly across the tested
ages, excluding the possibility of such interactions.

Readers may wonder to what extent our results are obscured by the
influence of experimental method, which is known to be sizable in the
cognitive developmental literature (({\textbf{???}}) et al., 2018).
Inspection of plots where effect sizes has been residualized from
experimental method (Supplementary figure SX) looks virtually identical.

\section{Discussion}\label{discussion}

Our meta-analysis synthesizes the available literature on infants'
preference for speech sounds. Our results confirm that infants reliably
prefer speech over other types of sounds from birth. When all studies
were considered together with no moderators, we found a sizable
intercept (g=.XX). For comparison, the main effect for native vowel
discrimination using looking time methods is estimated at .25 (Tsuji \&
Cristia, 2013; data inspected in metalab.stanford.edu on 2019-10-18). We
had predicted infants' speech preference to be larger when the
competitor was an artificial sound than when it was a natural one; when
the competitor was non-vocal; and when the speech was in the infants'
native language. In fact, we were unable to disprove the null hypothesis
of no difference for all three factors, with widely overlapping
distributions of effect sizes for studies varying along the three
dimensions.

We had also hypothesized age to play a major role, because it may
correlate with a reshaping of the category definition for speech itself.
Indeed, studies comparing processing of human speech against human
non-speech as well as animal vocalizations more generally (Vouloumanos
et al., 2010; Mac Donald et al., 2019) often discuss these age-related
differences in categorization of these sounds. Surprisingly, age did not
significantly moderate the overall preference for speech.

From birth and regardless of changes co-occurring with age, infants show
a preference for speech, which cannot be reduced to three simpler
explanations: naturalness, vocalness, or familiarity (represented here
by the native/foreign contrast). This capacity to preferentially listen
to speech sounds from birth suggests that infants are born with the
capacity to recognize their conspecifics' communication signals. This
parallels what has been proposed by the Conspec model for faces: infants
would be born with knowledge about faces, enabling them to orient their
attention toward them, even without any prior exposure to faces (Morton
\& Jonhson, 1991). The fact that familiarity with the language used in
the experiment did not modulate infants' preference suggest that
exposure did not play a crucial role for speech either. However,
contrary to faces, fetuses are exposed to speech that is low-pass
filtered by the womb throughout the last trimester of gestation (Querleu
et al., 1988, Lecanuet \& Granier-Deferre, 1993). It is therefore
possible that prenatal experience with low-pass filtered speech helps
infants to form a representation of speech, independently of the
language spoken.

The Conspec model proposes that faces would be detected because of their
spatial structure (Morton \& Johnson, 1991). Similarly, it is possible
that infants prefer speech because of its complex acoustic structure and
fast transitions (Rosen, 2007). Speech is characterized by joint
spectral and temporal modulations at specific rates (Singh \&
Theunissen, 2003). It is possible that infants attune to this specific
spectro-temporal structure (though see {\textbf{???}}, for evidence that
this explanation alone may not be sufficient for neural responses).
Testing this explanation would require to carry out acoustic analyses of
the actual stimuli used in the studies. Thus, we recommend interested
researchers to gather more data in which the competitor is acoustically
simple versus complex; and to deposit the actual stimuli in a public
archive such as the Open Science Framework (REF).

One may wonder whether we fail to find many differences because of a
lack of statistical power, particularly in view of between-study
variability. We think it is unlikely that all null results reported in
this paper are due to this. Inspection of results in Figure 3 show that
the confidence intervals of all conditions overlap almost exactly.
Moreover, Table XX shows that the estimate for all these factors is
close to zero (the maximum being XX). That said, more data would be
welcome to confirm our results with more statistical power. It would
also be important to carry out more tests on infants older than 9
months. Language production gains in complexity at about this age
{[}({\textbf{???}}), which could affect infants' speech preference. We
particularly recommend using as competitor natural vocal stimuli, and as
target foreign speech, which would help fill in an important gap in our
dataset.

Another finding of our meta-analysis is that the distribution of effect
sizes in the literature is consistent with publication bias, in view of
a strong asymmetry of the funnel plot. In fact, the trim-and-fill method
suggested XX points may be missing, which is a considerable number given
that we have XX effect sizes in total (i.e., a quarter more would be
missing). Unsurprisingly, the missing studies are in the negative
section, i.e., a preference \emph{against} speech, a result that could
lead authors to doubt their own data and not submit it to journals, or
that would be considered odd by reviewers and editors, who may ask that
the data be removed (or who may recommend the paper to be rejected
altogether). These missing studies constitute an important limitation of
our results. The literature being biased toward positive effect sizes,
the true effect size might be smaller than the one we found (vertical
line on Figure X).

Ultimately, preferential processing of speech may support higher level
cognitive tasks. The human species is a highly social one. Detecting
speech signals would allow to integrate it with other sensory percepts,
such as faces, to form multisensory representations of conspecifics
(({\textbf{???}}) et al., 2009). This would lay the track for social
cognition. Identifying speech signals and paying attention to them would
allow infants to form complex representations of the sensory world, that
they can manipulate cognitively. Infants could categorize visual stimuli
(i.e.~associate a label to a category of objects) when they were
associated to speech, but not pure tones or backward speech
(({\textbf{???}}) \& Waxman, 2007; ({\textbf{???}}) et al, 2010;
({\textbf{???}}) et al, 2013). Interestingly, infants categorized visual
stimuli when presented with speech, melodies, or monkey vocalizations
(Fulkerson \& Haaf, 2003; Ferry et al, 2013). These results support the
idea that infants may preferentially process complex sounds. Finally,
the preference itself may also be a meaningful index of processing that
can be used to identify children at risk (({\textbf{???}}) et al.,
2019). It is therefore important to take stock of what we know today.

Given the crucial importance of understanding infants' speech
preference, we make the following recommendations for further data
collection, analysis, and reporting. First, authors should strive to
increase their sample sizes. The median sample size at present is 20,
which is close to the field standard (({\textbf{???}}) et al., 2018) but
much lower than current recommendations (({\textbf{???}}), 2017).
Second, authors should consider proposing their studies as registered
reports (({\textbf{???}})). In this new publication scheme (available
for Developmental Science, Infancy, Infant Behavior and Development, and
Journal of Child Language at the time of writing, see a full up-to-date
list on \url{https://cos.io/rr/}), manuscripts are submitted before data
are collected. Reviewers and editors make publication decisions based
solely on the introduction and methods. Once the paper is accepted, the
author collects the data, analyses it according to a pipeline described
in the accepted methods, and writes up the rest of the manuscript. The
paper is then reviewed once more for readability, but it cannot be
rejected if the results are surprising or uncomfortable for the field.
Third, for authors who would rather not follow this publication route,
we still strongly recommend the use of pre-specified analysis plans.
These have the virtue of, when used correctly, allowing both authors and
readers to separate confirmation from exploration, reducing the
likelihood of inadvertently engaging in questionable research practices
({\textbf{???}}), known to increase false positives. Finally, we
strongly recommend reviewers and editors to evaluate submitted
manuscripts on the basis of the quality of the methods, and not of the
results. If a study fails to report a speech preference, or actually
reports a preference for the competitor, this may actually reflect the
reality of this phenomenon. For interested readers who intend to collect
such data, we recommend caution when designing the study, and
transparency when reporting it. Our dataset can be community-augmented,
and we invite researchers investigating this phenomenon to complement it
with any data they would have
(\href{https://osf.io/4stz9/?view_only=d0696591ebf34bfc8430f848cd945ca8}{Anonymized
link}), whatever the results and publication status.

\newpage

\begingroup
\setlength{\parindent}{-0.5in} \setlength{\leftskip}{0.5in}

\hypertarget{refs}{}

\endgroup


\end{document}
