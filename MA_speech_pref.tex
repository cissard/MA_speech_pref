% Options for packages loaded elsewhere
\PassOptionsToPackage{unicode}{hyperref}
\PassOptionsToPackage{hyphens}{url}
%
\documentclass[
  english,
  man]{apa6}
\usepackage{amsmath,amssymb}
\usepackage{lmodern}
\usepackage{ifxetex,ifluatex}
\ifnum 0\ifxetex 1\fi\ifluatex 1\fi=0 % if pdftex
  \usepackage[T1]{fontenc}
  \usepackage[utf8]{inputenc}
  \usepackage{textcomp} % provide euro and other symbols
\else % if luatex or xetex
  \usepackage{unicode-math}
  \defaultfontfeatures{Scale=MatchLowercase}
  \defaultfontfeatures[\rmfamily]{Ligatures=TeX,Scale=1}
\fi
% Use upquote if available, for straight quotes in verbatim environments
\IfFileExists{upquote.sty}{\usepackage{upquote}}{}
\IfFileExists{microtype.sty}{% use microtype if available
  \usepackage[]{microtype}
  \UseMicrotypeSet[protrusion]{basicmath} % disable protrusion for tt fonts
}{}
\makeatletter
\@ifundefined{KOMAClassName}{% if non-KOMA class
  \IfFileExists{parskip.sty}{%
    \usepackage{parskip}
  }{% else
    \setlength{\parindent}{0pt}
    \setlength{\parskip}{6pt plus 2pt minus 1pt}}
}{% if KOMA class
  \KOMAoptions{parskip=half}}
\makeatother
\usepackage{xcolor}
\IfFileExists{xurl.sty}{\usepackage{xurl}}{} % add URL line breaks if available
\IfFileExists{bookmark.sty}{\usepackage{bookmark}}{\usepackage{hyperref}}
\hypersetup{
  pdftitle={Infants' preference for speech is stable across the first year of life: Meta-analytic evidence},
  pdfauthor={Cécile Issard1, Sho Tsuji2, \& Alejandrina Cristia1},
  pdflang={en-EN},
  pdfkeywords={Meta-analysis, Infants, Speech preference, Conspecific detection, Developmental tuning},
  hidelinks,
  pdfcreator={LaTeX via pandoc}}
\urlstyle{same} % disable monospaced font for URLs
\usepackage{graphicx}
\makeatletter
\def\maxwidth{\ifdim\Gin@nat@width>\linewidth\linewidth\else\Gin@nat@width\fi}
\def\maxheight{\ifdim\Gin@nat@height>\textheight\textheight\else\Gin@nat@height\fi}
\makeatother
% Scale images if necessary, so that they will not overflow the page
% margins by default, and it is still possible to overwrite the defaults
% using explicit options in \includegraphics[width, height, ...]{}
\setkeys{Gin}{width=\maxwidth,height=\maxheight,keepaspectratio}
% Set default figure placement to htbp
\makeatletter
\def\fps@figure{htbp}
\makeatother
\setlength{\emergencystretch}{3em} % prevent overfull lines
\providecommand{\tightlist}{%
  \setlength{\itemsep}{0pt}\setlength{\parskip}{0pt}}
\setcounter{secnumdepth}{-\maxdimen} % remove section numbering
% Make \paragraph and \subparagraph free-standing
\ifx\paragraph\undefined\else
  \let\oldparagraph\paragraph
  \renewcommand{\paragraph}[1]{\oldparagraph{#1}\mbox{}}
\fi
\ifx\subparagraph\undefined\else
  \let\oldsubparagraph\subparagraph
  \renewcommand{\subparagraph}[1]{\oldsubparagraph{#1}\mbox{}}
\fi
% Manuscript styling
\usepackage{upgreek}
\captionsetup{font=singlespacing,justification=justified}

% Table formatting
\usepackage{longtable}
\usepackage{lscape}
% \usepackage[counterclockwise]{rotating}   % Landscape page setup for large tables
\usepackage{multirow}		% Table styling
\usepackage{tabularx}		% Control Column width
\usepackage[flushleft]{threeparttable}	% Allows for three part tables with a specified notes section
\usepackage{threeparttablex}            % Lets threeparttable work with longtable

% Create new environments so endfloat can handle them
% \newenvironment{ltable}
%   {\begin{landscape}\centering\begin{threeparttable}}
%   {\end{threeparttable}\end{landscape}}
\newenvironment{lltable}{\begin{landscape}\centering\begin{ThreePartTable}}{\end{ThreePartTable}\end{landscape}}

% Enables adjusting longtable caption width to table width
% Solution found at http://golatex.de/longtable-mit-caption-so-breit-wie-die-tabelle-t15767.html
\makeatletter
\newcommand\LastLTentrywidth{1em}
\newlength\longtablewidth
\setlength{\longtablewidth}{1in}
\newcommand{\getlongtablewidth}{\begingroup \ifcsname LT@\roman{LT@tables}\endcsname \global\longtablewidth=0pt \renewcommand{\LT@entry}[2]{\global\advance\longtablewidth by ##2\relax\gdef\LastLTentrywidth{##2}}\@nameuse{LT@\roman{LT@tables}} \fi \endgroup}

% \setlength{\parindent}{0.5in}
% \setlength{\parskip}{0pt plus 0pt minus 0pt}

% \usepackage{etoolbox}
\makeatletter
\patchcmd{\HyOrg@maketitle}
  {\section{\normalfont\normalsize\abstractname}}
  {\section*{\normalfont\normalsize\abstractname}}
  {}{\typeout{Failed to patch abstract.}}
\patchcmd{\HyOrg@maketitle}
  {\section{\protect\normalfont{\@title}}}
  {\section*{\protect\normalfont{\@title}}}
  {}{\typeout{Failed to patch title.}}
\makeatother
\shorttitle{Preference for speech sounds in infancy}
\keywords{Meta-analysis, Infants, Speech preference, Conspecific detection, Developmental tuning\newline\indent Word count: 6257}
\DeclareDelayedFloatFlavor{ThreePartTable}{table}
\DeclareDelayedFloatFlavor{lltable}{table}
\DeclareDelayedFloatFlavor*{longtable}{table}
\makeatletter
\renewcommand{\efloat@iwrite}[1]{\immediate\expandafter\protected@write\csname efloat@post#1\endcsname{}}
\makeatother
\usepackage{csquotes}
\ifxetex
  % Load polyglossia as late as possible: uses bidi with RTL langages (e.g. Hebrew, Arabic)
  \usepackage{polyglossia}
  \setmainlanguage[]{english}
\else
  \usepackage[main=english]{babel}
% get rid of language-specific shorthands (see #6817):
\let\LanguageShortHands\languageshorthands
\def\languageshorthands#1{}
\fi
\ifluatex
  \usepackage{selnolig}  % disable illegal ligatures
\fi
\newlength{\cslhangindent}
\setlength{\cslhangindent}{1.5em}
\newlength{\csllabelwidth}
\setlength{\csllabelwidth}{3em}
\newenvironment{CSLReferences}[2] % #1 hanging-ident, #2 entry spacing
 {% don't indent paragraphs
  \setlength{\parindent}{0pt}
  % turn on hanging indent if param 1 is 1
  \ifodd #1 \everypar{\setlength{\hangindent}{\cslhangindent}}\ignorespaces\fi
  % set entry spacing
  \ifnum #2 > 0
  \setlength{\parskip}{#2\baselineskip}
  \fi
 }%
 {}
\usepackage{calc}
\newcommand{\CSLBlock}[1]{#1\hfill\break}
\newcommand{\CSLLeftMargin}[1]{\parbox[t]{\csllabelwidth}{#1}}
\newcommand{\CSLRightInline}[1]{\parbox[t]{\linewidth - \csllabelwidth}{#1}\break}
\newcommand{\CSLIndent}[1]{\hspace{\cslhangindent}#1}

\title{Infants' preference for speech is stable across the first year of life: Meta-analytic evidence}
\author{Cécile Issard\textsuperscript{1}, Sho Tsuji\textsuperscript{2}, \& Alejandrina Cristia\textsuperscript{1}}
\date{}


\authornote{

This work was supported by a Fyssen Foundation Post-doctoral Fellowship to C.I., an Agence Nationale de la Recherche grant to A.C. (ANR-17-CE28-0007 LangAge, ANR-16-DATA-0004 ACLEW, ANR-14-CE30-0003 MechELex, ANR-17-EURE-0017); and the J. S. McDonnell Foundation Understanding Human Cognition Scholar Award to A.C.

The authors declare no conflict of interest. Funding sources did not take part in study design, data collection or analysis.

Our data is fully available in the corresponding OSF repository: \url{http://tidy.ws/bqjc4U}

Correspondence concerning this article should be addressed to Cécile Issard, Laboratoire de Sciences Cognitives et Psycholinguistique, Département d'Études Cognitives, Ecole Normale Supérieure, 29 rue d'Ulm, 75005 Paris, France. E-mail: cecile.issard gmail.com

}

\affiliation{\vspace{0.5cm}\textsuperscript{1} Laboratoire de Sciences Cognitives et Psycholinguistique, Ecole Normale Supérieure, Département d'Études Cognitives\\\textsuperscript{2} International Research Center for Neurointelligence, The University of Tokyo}

\abstract{
Previous work suggested that humans' sophisticated speech perception abilities stem from an early capacity to pay attention to speech in the auditory environment. Previous studies have therefore tested if infants prefer speech to other sounds at a variety of ages, but provided contrasted results. In this paper, we make the hypothesis that speech is initially encoded similarly to other natural or vocal sounds, and that infants tune to speech during the first year of life as they acquire their native language. To test this hypothesis, we conducted a meta-analysis of experiments testing speech preference in infants, sorting experiments by whether they used native or foreign speech on the one hand, and vocal or non-vocal, natural or artificial sound on the other hand. Synthesizing data from 791 infants across 39 experiments, we found a medium effect size, confirming at the scale of the literature that infants reliably prefer speech over other sounds. However, this preference was not significantly moderated by the language used, nor vocal quality, or naturalness of the competitor. Strikingly, we found no effect of age: infants showed the same strength of preference throughout the first year of life. Speech therefore appears to be preferred from birth, even to other natural or vocal sounds. These results suggest that speech is processed in a specific way by an innate dedicated system, disctinct from other sounds processing.
}



\begin{document}
\maketitle

\hypertarget{significance}{%
\section{Significance}\label{significance}}

\begin{itemize}
\tightlist
\item
  Infants reliably prefer natural speech over other types of sounds
\item
  Preference is stable from birth to the end of the first year of life
\item
  Speech is preferred over both artificial and other natural sounds
\item
  Speech is preferred over both non-vocal and other vocal sounds
\item
  The language used for the speech sounds made no significant difference
\end{itemize}

\hypertarget{introduction}{%
\section{Introduction}\label{introduction}}

\textbf{ci2all This first paragraph summarizes the main theory of speech preference, with the factors most commonly discussed (human/monkeys, speech/other vocalizations). I hope this will make it clear that we do know this view, and that our MA nuances it by including less cited/overlooked papers (not to mention that these mostly cited papers are all from the Vouloumanos lab. It might be good to take into account results from other labs).}

\textbf{ac2ci I made some edits particularly at the end of the para}

Humans acquire their communication skills from early infancy, with specialized speech perception abilities well before they produce their first word. These perceptual capacities manifest in an early preference for speech over other types of sound (Ecklund-Flores \& Turkewitz, 1996; Vouloumanos, Hauser, Werker, \& Martin, 2010; Vouloumanos \& Werker, 2007). Some studies show, at birth, a preference for speech over sine-wave speech {[}i.e., time-varying sinusoidals tracking the fundamental frequency and the first three formants; Vouloumanos and Werker (2007){]}. Studies find no preference between speech and monkey calls at birth (Shultz \& Vouloumanos, 2010; Vouloumanos, Hauser, Werker, \& Martin, 2010), but a preference for speech over monkey calls at three month old (Vouloumanos, Hauser, Werker, \& Martin, 2010). Results like these have led to the theoretical proposal that speech is a privileged signal for humans, whereby newborns have a preference for the vocalizations of humans and non-human primates, and by three months tune in to human speech specifically (Vouloumanos \& Waxman, 2014). These studies even found that three-month-olds favor human speech over other human vocal sounds, such as coughing (Shultz \& Vouloumanos, 2010). Because these results have been obtained both with the participants' native language (Vouloumanos, Hauser, Werker, \& Martin, 2010; Vouloumanos \& Werker, 2007), and foreign language (Shultz \& Vouloumanos, 2010), it was claimed that infants tune to the speech signal itself, and not simply the familiar sounds of their native language (Vouloumanos \& Waxman, 2014). There have been variants of such views. For instance, Ecklund-Flores and Turkewitz (1996) documented, at birth, behavioral responses that seemed specific to speech, as compared to low-pass or band-pass filtered speech, and other natural sounds. This suggested that these behavioral responses were driven by information carried below 3500 Hz, related to characteristics of speech. The specificity of this behavioral response at a young age seems to suggest an early primacy of the speech bias.

After decades of theoretical and empirical work in this area, the field has converged in a series of questions about the mechanisms that support speech perception in infants, not all of which have answers already: Does the auditory system process all sounds similarly, or are there dedicated mechanisms for speech? Does the preference for speech emerge from a low-level bias, whereby the auditory system better encodes sounds with acoustic properties that match particularly well with the acoustics of speech (processing similarly well other natural and vocal sounds)? Or does the preference stem from high-level linguistic processing (i.e., recognition of properties like syllabic structure)? In addition to asking these questions regarding preference at birth, drawing a developmental curve is crucial to tackle these questions: If speech is processed in a specific way as compared to other sounds, then from birth, infants should show a preference for speech as compared to other sounds. If speech benefits from properties that also apply to other sounds, then infants should first have a general preference for acoustically close sounds that match these properties (among them speech), and then gradually tune to speech as they are exposed to speech. Finally, if speech processing relies on the same auditory-general mechanisms as any other sound, then speech preference should emerge during the first year of life with language acquisition and depend on the language learned. \textbf{ac2all I like the questions at the beginning of the paragraph, but not the hypotheses/predictions starting from the developmental curve part. We'll explain this a lot more clearly after we introduce the distinct hypotheses below. I'd prefer to add a question like ``And how does this change with infant development (including exposure to one or more languages)?'' and then I'd remove the para break and change the next sentence to start with ``To seek the state-of-the-art answers to these questions, we synthesize\ldots{}''}

Here, to test these hypotheses, we synthesize the available empirical data on infants' preferences for speech over non-speech sounds from birth to the end of the first year of life using a meta-analytic approach. In order to do so, we clarify the conceptual separation between three factors that could in principle explain speech preference, but have not often been spelled out as alternative hypotheses in the past {[}but see XXX, p.~YY, and ZZ, p.~QQ, among other for pre-figurations of these ideas{]}.
\textbf{ac2ci factors not yet introduced, so I changed the wording. Also, do you think it's strategically better to say that these ideas have not been spelled out \& cite some people? or can we just cite people who conceptually separated them? Also, could we cite people who have talked about each of the biases below, even if in a way that was more confused across factors?}

\begin{figure}
\includegraphics[width=5.31in]{figures_intro/tuning} \caption{Developmental tuning for speech in the first year of life: infant would first discriminate artificial (purple) and natural (green) sounds, then vocal sounds (orange) within natural sounds, and ultimately speech as a separate category within natural and vocal sounds.}\label{fig:venn}
\end{figure}

\textbf{ac2ci in each of the following paragraphs there was an assumption of increases with exposure that is not necessary for each of the hypotheses. I removed it throughout! I also changed the order as Sho had suggested}

A first possibility, which we will call a \textbf{naturalness preference}, to explain infants' preference for speech is that the auditory system is tailored for the acoustic properties of speech, as well as sounds with close acoustical properties (e.g., other natural or vocal sounds). This idea has been pervasive in the field of auditory neuroscience, relying on the hypothesis that the auditory system has been shaped by evolutionary pressure to efficiently encode ecologically important sounds, typically environmental sounds signalling a threat, or conspecific vocalizations, and more generally natural sounds such as wind, rain, or the sound of a river, as well as those produced by biological systems, such as heart beats, step sounds, or animal vocalizations, and speech. Following this hypothesis, numerous studies have shown that natural sounds are processed more efficiently by the auditory system, from the cochlea (Smith \& Lewicki, 2006) to the auditory cortex (see Mizrahi, Shalev, \& Nelken, 2014 for a review). If a naturalness bias (partially) explains infants' speech preference, then we should observe a stronger preference at birth for speech over artificial competitors than when contrasting speech against another natural sounds.
\textbf{ac2ci the last sentence doesn't follow, right? this hypothesis doesn't speak of dev't. I removed it. I also changed the preceding sentence to the prediction I think we should make}

A second possibility, the \textbf{vocal quality preference}, is that auditory coding is tailored to vocal sounds more specifically, rather than to all natural sounds. Vocal sounds have acoustic signatures, since they are characterized by modulations introduced by the vocal tract, with harmonically related energy peaks. Animal vocalizations (among them speech) are characterized by low temporal and spectral modulations, therefore form a separate acoustic category within natural sounds (Singh \& Theunissen, 2003). This view is supported by studies that found that newborns showed no preference for speech compared to monkey calls, which are both vocal sounds (Vouloumanos, Hauser, Werker, \& Martin, 2010). If a vocal bias explains at least partially speech preferences, then we should observe a stronger preference for speech over non-vocal competitors than when contrasting speech against other vocal sounds.
\textbf{ac2ci a sentence in the middle \& end doesn't follow, because this hypothesis doesn't speak of dev't. I removed them. I also changed the Vouloumanos cite to fit the explanation}

A third distinct conceptual hypothesis is that there are dedicated perceptual mechanisms that are specific to speech, and distinct from how other vocal or natural sounds are processed -- \textbf{a speech preference}. Previous studies provided evidence for this hypothesis: newborns respond more to natural speech than to heartbeat, high-pass or band-ass filtered speech (Ecklund-Flores \& Turkewitz, 1996). Newborn and one-month-old infants even prefer natural to low-pass filtered speech, which mimics what's heard in the womb (Cooper \& Aslin, 1994; Ecklund-Flores \& Turkewitz, 1996). If a speech bias explains speech preference at least partially, we should see preferences for speech over other sounds, including natural and vocal sounds.

A final possibility would be that the fetus' brain initially encodes all sounds equally well, and the preference emerges due to exposure to speech -- a \textbf{familiarity preference}. Previous results about language preference make this explanation plausible: Newborns prefer their native speech as compared to prosodically distinct foreign speech (e.g., Mehler et al., 1988; Moon, Cooper, \& Fifer, 1993). If one could test unborn children before exposure to much speech, speech would not be preferred to any other sound. Since that is impossible, given that newborns have been exposed to speech for three months by fullterm birth, one predicts from this hypothesis that preferences would get stronger during the first year of life, and that speech preference would be greater for the infant's native language than foreign languages.
\textbf{ac2ci I made lots of changes in this paragraph, in part to clarify the explanation and avoid repetition of predictions}

\hypertarget{a-meta-analytic-approach}{%
\subsection{A meta-analytic approach}\label{a-meta-analytic-approach}}

In sum, previous results on infants' preference for speech are broadly compatible with a preference for natural over artificial, vocal over non-vocal, speech over non-speech, and/or familiar over unfamiliar sounds, potentially interacting with infants' age. It should be noted that all four preferences could be at play, and indeed several of them are sometimes integrated in one and the same theoretical framework {[}CITE CITE CITE CITE CITE CITE CITE{]}. However, from a conceptual standpoint, they reflect different hypotheses and assumptions about infants' perceptual system, and thus it is worthwhile to entertain them as separate accounts.

In fact, with the right data, their predictions can be teased apart, and with enough power, the field would be able to attribute relative weights to each of the biases. However, if the different dimensions are typically confounded in empirical work, then theoretical proposals building on this research do not stand on solid ground.

A first step in this direction is to synthesize previous literature, to assess the extent to which these conceptually distinct factors have been independently manipulated in experimental designs. We can then measure the extent to which the literature supports their joint predictions (i.e., a preference for speech), and their individual predictions, as far as possible given the extent of currently confounded factors in past literature. This also allows us to identify empirical gaps (e.g., age groups that are under-represented but are crucial to tease apart two factors) and conceptual gaps (e.g., use of stimuli that systematically confounds two or more of those explanations).

In this paper, we seek to directly test these potential mechanisms by employing a meta-analytic approach, which is recommended over narrative reviews (\textbf{cristia\_meta-analytic\_2021?}). A meta-analysis can integrate data from experiments that vary in their methodology, as well as test the effect of factors of theoretical importance, by redescribing the stimuli used as a function of those factors. For instance, a study measuring preference for native speech over white noise provides data on a natural versus artificial contrast, as well as a vocal versus non-vocal contrast, thus accounting for how the same stimuli can be both natural and vocal or artificial and non-vocal.

Also, we can draw a developmental timeline across the age range covered by the literature, beyond age groups tested within papers. This is particularly useful in developmental psychology, which relies on age-related differences that need to be tested statistically (Gelman \& Stern, 2006). Meta-analyses offer a powerful statistical approach to directly test for interactions with age across the whole age-range covered by the literature. To give an example from a previous developmental meta-analysis, it had been proposed that infants' preference for novel or familiar items related to infants' age such that, all things equal, younger infants showed familiarity preferences whereas older infants exhibited novelty preferences (Hunter \& Ames, 1988). However, stable familiarity preferences across the first two years have been found for word segmentation in natural speech (Bergmann \& Cristia, 2016); and a stable novelty effect ensues for artificial grammars implemented in synthesized speech, whereas those implemented in natural speech led to stable familiarity preferences (Black \& Bergmann, 2017). Meta-analyses are therefore important to statistically and systematically test the theoretical predictions proposed in qualitative reviews, and show subtle effects that are difficult to see when reading the literature with a human eye.

Single experiments tell us about what a specific group of participants, presented with a specific set of stimuli, at a specific point in time has done. Meta-analyses are the following step because they provide a principled statistical approach to integrate those individual and specific results into a larger picture. By aggregating the numerous individual studies of a literature, meta-analyses gain statistical power. As a result, meta-analyses can reveal small effects that are difficult to show in individual experiments. By integrating data across different laboratories, they provide evidence for the generalizability of effects, and facilitate comparisons between experimental results.

Finally, meta-analyses offer tools to detect publication bias in the literature. By aggregating all the available evidence for a phenomenon, we can see if the distribution of effect sizes has an unexpected shape, typically with an excess of positive results due to the difficulty to publish null or negative results. We can further integrate this information, and derive a new estimate of the overall effect size.

\hypertarget{the-present-study}{%
\subsection{The present study}\label{the-present-study}}

Meta-analyses provide a unique vantage point on a body of work as a whole. We therefore first check for how strong infants' preference for speech over other types of sounds is according to the public body of literature. We additionally assess this body of data for evidence of publication bias.

We then turn to our key interest, namely shedding light on the potential mechanisms underlying infants' speech preferences. Meta-regressions assess whether the proposed mechanisms of naturalness, vocal quality, and familiarity drive this preference, and how the preference develops over the first year of life -- within the limits of what previous work has been able to establish. Assuming all four mechanisms are at play, we predicted that infants will show (see Figure \ref{fig:hyp}):

\begin{enumerate}
\def\labelenumi{\arabic{enumi}.}
\tightlist
\item
  a greater preference for speech over other natural sounds as a function of age, but a preference for speech over artificial sounds that is stable over development;
\item
  a greater preference for speech over other vocal sounds as a function of age, but a stable preference for speech over non-vocal sounds over development;
\item
  a greater preference for native speech over non-speech as a function of age, but a smaller preference for foreign speech over non-speech with age.
\end{enumerate}

\textbf{ac2ci this doesn't fit in with Sho's proposal, but when I got here I realized -- what she's suggesting is not testable, right??}

\begin{figure}
\includegraphics[width=6.63in]{figures_intro/hypotheses} \caption{Hypothesized pattern of preference: the x axis shows age, the y axis represents the effect size derived from the contrast between a speech condition and a competitor condition (preference for speech over the competitor is plotted up; the lower quadrants are empty because we do not predict a preference for the competitor over speech). A: Speech contrasted to natural (green) or artificial (purple) competitors. B: Speech contrasted to vocal (orange) or non-vocal (cyan) competitors. C: Collapsing across competitors, separating speech in a foreign language (red); speech in the native language (blue).}\label{fig:hyp}
\end{figure}

\hypertarget{methods}{%
\section{Methods}\label{methods}}

This meta-analysis was carried out following PRISMA recommendations (Moher, Liberati, Tetzlaff, Altman, \& Group, 2009). In addition, we provide information on all steps (including PRISMA checklist, data, and code) for full transparency and accountability via online supplementary materials; \url{https://osf.io/4stz9/?view_only=d0696591ebf34bfc8430f848cd945ca8}.

\hypertarget{literature-search}{%
\subsection{Literature search}\label{literature-search}}

We composed the initial list of papers with suggestions by experts (authors of this work); one google scholar search (\emph{(``speech preference'' OR ``own-species vocalization'') AND infant - ``infant-directed''}), the same search in PubMed and PsycInfo (last searched on 2019-09-24); and a google alert. We also inspected the reference lists of all included papers. Finally, we emailed a major mailing list to ask for missing data. We received two replies, one of which revealed a formerly undiscovered published study, and communicated unpublished data (Santolin, Zettersten, \& Saffran, 2020).

\hypertarget{inclusion-criteria}{%
\subsection{Inclusion criteria}\label{inclusion-criteria}}

After a first screening based on titles and abstracts using more liberal inclusion criteria, we decided on final inclusion based on full paper reading. We included experiments that tested human infants from birth to one year of age, and contrasted speech sounds with any other type of sound, measuring behavioral preferences to the sounds (e.g., looking times). If a paper reported results from neurotypical and at-risk infants, we included only the data from the neurotypical group.

Given our key interest in the preference for speech over other sounds, we excluded studies that contrasted two different speech sounds (e.g., foreign vs.~native language, or adult vs.~child-directed speech, or mother vs.~stranger's voice); or two different non-speech sounds (e.g., backward speech vs.~animal vocalizations). In addition, we excluded experiments where the contrast presented to the infants could not be coded according to our three mechanistic explanations. This meant the exclusion of experiments where speech was presented in the mother's voice (which thus confounds between speech and individual voice recognition for our familiarity factor). Finally, we excluded neuroimaging experiments to avoid mixing results from different brain regions with different response profiles. We included published (i.e., journal articles) as well as unpublished works (i.e., doctoral dissertations) as long as sufficient information was provided.

A PRISMA flow chart summarizes the literature review and selection process (Figure \ref{fig:prisma}). The full list of the papers that were inspected together with final inclusion decisions are available in a decision spreadsheet (see the online supplementary materials; \url{https://osf.io/4stz9/?view_only=d0696591ebf34bfc8430f848cd945ca8}).

\begin{figure}
\includegraphics[width=3.2in]{figures_intro/PRISMA} \caption{PRISMA flowchart summarizing the literature review and selection process.}\label{fig:prisma}
\end{figure}

\hypertarget{coding}{%
\subsection{Coding}\label{coding}}

Data were coded by the first author. In addition, 20\% of the papers were randomly selected to be coded by the last author independently, with disagreements resolved by discussion. There were 10 disagreements out of a total of 260 fields filled in, and they were indicative of the coders not following the codebook, which led to a revision of all data in four variables.

The critical variables for our purpose are key stimuli characteristics, infant age, and testing method (central fixation, high amplitude sucking, head-turn preference procedure). As for key stimuli characteristics, we coded familiarity, naturalness, and vocal quality, as follows.

For \textbf{naturalness}, the competitor sound was coded as natural if it was produced by a biological organism without any further acoustic manipulation. Natural competitors included animal calls, environmental sounds (e.g.~wind or water sounds), heartbeat, bird song, non-speech vocalizations (e.g.~laughter or coughs). If the authors applied acoustic manipulations, the competitor was coded as artificial. Artificial competitors included sine-wave speech, filtered speech\footnote{In the case of filtered speech, the modulations introduced by the vocal tract are still present at the retained frequencies, and formant transitions are consistent with vocal production constraints. For this reason, filtered speech can be considered as vocal but not natural. Because the womb acts as a low-pass filter, newborn infants are familiar with low-pass filtered speech, but this familiarity fades after birth.}, white noise, instrumental music, and speech with altered rhythmic structure. The only exception was for newborn experiments presenting low-pass filtered speech mimicking the filtering applied by the womb. Given the recency of the intra-uterine environment to newborns (about 2 days), we coded these as natural.

For \textbf{vocal quality}, the competitor sound was considered as vocal if it was produced by an animal vocal tract (human or not), either original or modified. Vocal competitors included non-speech vocalizations, animal calls, bird songs, and filtered speech. Non-vocal competitors included backward speech (that has abrupt closures that cannot be produced by the vocal tract), white-noise, environmental sounds, instrumental music, heartbeat, and sine-wave speech (that lacks the harmonic structure introduced by the natural resonance of the vocal tract).

For \textbf{familiarity}, we considered the language in which the speech sounds were recorded (native or foreign).

We coded all the statistical information reported in the included papers. If reported, we coded the mean score and the standard deviation for speech, and the other sound separately. When infant-level data was provided, we recomputed the respective mean scores and standard deviations based on the reported individual scores. If reported, we also coded the t-statistic between the two sound conditions, or an F-statistic provided this was a two-way comparison. If effect sizes were directly reported as a Cohen's d or a Hedges' g, we also coded this.

\hypertarget{effect-sizes}{%
\subsection{Effect sizes}\label{effect-sizes}}

Once the data were coded, we extracted effect sizes, along with their respective variance. Effect sizes were standardized differences (Cohen's d) between response to speech vs.~the competitor.
If effect sizes were not directly reported in the papers, we computed them using the respective means and SDs (Lipsey \& Wilson, 2001), or a t- or F-statistic (Dunlap, Cortina, Vaslow, \& Burke, 1996). As our effect sizes came from within-subject comparisons (e.g., looking time of the same infant during speech and monkey calls), we needed to take into account the correlation between the two measurements in effect sizes and effect size variances computations. We computed this correlation based on the t-statistic, the respective means, and SDs (Lipsey \& Wilson, 2001) if they were all reported; or imputed this correlation randomly if not. We finally calculated the variance of each effect size (Lipsey \& Wilson, 2001). Cohen's d were transformed to Hedges' g by multiplying d by a correction for small sample sizes based on the degree of freedom (Borenstein, Hedges, Higgins, \& Rothstein, 2011).

We did not center age because our hypotheses included a developmental progression from birth to the end of the first year of life. We were therefore interested in the intercept at age 0 (i.e., birth).

Analyses use the R (R Core Team, 2018) package Robumeta (Hedges, Tipton, \& Johnson, 2010), which allows us to fit meta-analytic regressions that take into account the correlated structure of the data when repeated measures are obtained from the same infant groups within papers.

\hypertarget{results}{%
\section{Results}\label{results}}

\hypertarget{database-description}{%
\subsection{Database description}\label{database-description}}

We found a total of 19 publications (labeled with an asterisk in the reference list) reporting 39 experiments, for a total of 791 infants, and 55 (not mutually independent) effect sizes, see Figure \ref{fig:forest}. 16 papers have been submitted to or published in peer-reviewed journals (Colombo \& Bundy, 1981; Cooper \& Aslin, 1994; Curtin \& Vouloumanos, 2013; Ecklund-Flores \& Turkewitz, 1996; Santolin, Russo, Calignano, Saffran, \& Valenza, 2019; Segal \& Kishon-Rabin, 2011; Segal, Kligler, \& Kishon-Rabin, 2021; Shultz \& Vouloumanos, 2010; Sorcinelli, Ference, Curtin, \& Vouloumanos, 2019; Spence \& DeCasper, 1987; Vanden Bosch der Nederlanden \& Vouloumanos, 2021; Vouloumanos \& Curtin, 2014; Vouloumanos, Druhen, Hauser, \& Huizink, 2009; Vouloumanos, Hauser, Werker, \& Martin, 2010, 2010; Vouloumanos \& Werker, 2004; Vouloumanos \& Werker, 2007; Yamashiro, Curtin, \& Vouloumanos, 2020). The remaining 1 publication contributing 8 effect size was a thesis (Ference, 2018). 6 more effect sizes were contributed by authors of unpublished work (Santolin, Zettersten, \& Saffran, 2020).

Experiments tended to have small sample sizes, with a median N of 16 children (Range = {[}60, 4{]}, M = 19.76), which is close to the field standard (Bergmann et al., 2018), but much lower than current recommendations (Oakes, 2017). Infants ranged from 0 to 12 months (1.50 to 380.50 days), although the majority were under 6 months of age (61.54\% of the experiments). Individual samples comprised 46\% of female participants on average. Infants were native of 6 different languages across the whole database (English, French, Russian, Yiddish, Hebrew, Italian).
Experiments were performed in 10 different laboratories from 4 different countries (United States, Canada, Israel, Italy). 3 experimental methods were used: 79.49\% of the experiments used Central Fixation (CF) (also called sequential looking preference procedure) (Colombo \& Bundy, 1981; Cooper \& Aslin, 1994; Curtin \& Vouloumanos, 2013; Ference, 2018; Santolin, Russo, Calignano, Saffran, \& Valenza, 2019; Santolin, Zettersten, \& Saffran, 2020; Segal \& Kishon-Rabin, 2011; Segal, Kligler, \& Kishon-Rabin, 2021; Shultz \& Vouloumanos, 2010; Sorcinelli, Ference, Curtin, \& Vouloumanos, 2019; Vanden Bosch der Nederlanden \& Vouloumanos, 2021; Vouloumanos \& Curtin, 2014; Vouloumanos, Druhen, Hauser, \& Huizink, 2009; Vouloumanos, Hauser, Werker, \& Martin, 2010; Vouloumanos \& Werker, 2004; Yamashiro, Curtin, \& Vouloumanos, 2020); 7.69\% used High-Amplitude Sucking (HAS) (Spence \& DeCasper, 1987; Vouloumanos, Hauser, Werker, \& Martin, 2010; Vouloumanos \& Werker, 2007); and 12.82\% used Head-turn Preference Procedure (HPP) (Ecklund-Flores \& Turkewitz, 1996). Trial length was fixed in 20.51\% of the experiments, and infant-controlled in 76.92\% of the experiments.

Speech sounds were spoken by a female in 94.87\% of the experiments, with an infant-directed prosody in 53.85\% of the experiments. Speech was presented in isolated segments (i.e.~words or syllables) in 10.26\% of the experiments, and full sentences or passages in 48.72\% of the experiments. Speech stimuli were recorded in the infant native language in 64.10\% of the experiments. Strikingly, experiments using the infants' native language tested infants from 0 to 12 months of age, whereas experiments using a foreign language only tested infants from 3 to 9 months of age (see Figure \ref{fig:lang}).
The competitor sound was vocal in 46.15\% of the experiments. The competitor sound was natural 46.15\% of the experiments. Stimuli were both vocal and natural in 35.90\% of the experiments.
The stimuli distributions across the database are summarized on Figures \ref{fig:stimuli} and \ref{fig:competitors}.

\begin{figure}
\centering
\includegraphics{MA_speech_pref_files/figure-latex/stimuli-1.pdf}
\caption{\label{fig:stimuli}Histograms of the number of effect sizes for each language and moderator status.}
\end{figure}

\begin{figure}
\centering
\includegraphics{MA_speech_pref_files/figure-latex/competitors-1.pdf}
\caption{\label{fig:competitors}Histogram of the number of effect sizes for each competitor.}
\end{figure}

\hypertarget{average-effect-size}{%
\subsection{Average effect size}\label{average-effect-size}}

\begin{figure}
\centering
\includegraphics{MA_speech_pref_files/figure-latex/forest-1.pdf}
\caption{\label{fig:forest}Forest plot of effect sizes available in the literature, along with their respective moderator status. The average effect size is plotted on the bottom line.}
\end{figure}

We integrated all effect sizes in a meta-analytic regression without any moderator, and found an average effect size g of 0.40 (SE = 0.07, CI = {[}0.26 , 0.55{]}) (Table 1, and Figure \ref{fig:forest}, diamond), corresponding to a medium effect size.
Heterogeneity among effect sizes was estimated at \(\tau^2\) = 0.16 (I\textsuperscript{2} = 78.38\%), which was significant (Q = 206.59, p \textless{} 0.01) despite the removal of outliers before running the model. This strongly suggest differences across experiments, and invites analyses using moderators.

\begin{table}[tbp]

\begin{center}
\begin{threeparttable}

\caption{\label{tab:Table1}Statistical results of meta-regression without any moderator.}

\begin{tabular}{lcccc}
\toprule
 & estimate & SE & t & confidence interval\\
\midrule
average effect size & 0.40 & 0.07 & 5.59 & 0.26 - 0.55\\
\bottomrule
\end{tabular}

\end{threeparttable}
\end{center}

\end{table}

\hypertarget{publication-bias}{%
\subsection{Publication bias}\label{publication-bias}}

\begin{figure}
\centering
\includegraphics{MA_speech_pref_files/figure-latex/bias-1.pdf}
\caption{\label{fig:bias}Funnel plot of effect sizes and their respective standard errors. Black dots: effect sizes observed in the literature. White dots: missing effect sizes, suggestive of a publication bias\(^2\). Vertical line: average effect size after filling the missing effect sizes.}
\end{figure}

We checked for the presence of a potential publication bias in the body of literature by studying the relationship between standard errors of effect sizes as a function of Hedges' g (see funnel plot in Figure \ref{fig:bias})\footnote{If the literature is not biased, effect sizes should be evenly distributed around the mean effect size, with increasing standard error as they go away from the mean effect size (both in the positive and negative directions, white triangle in the funnel plot). This is reflected by a symmetrical funnel plot, with no linear relationship between effect sizes and standard errors.}. A regression test on these data was significant (z = 6.70, p \textless{} 0.01), as was the Kendall's tau rank correlation test for funnel plot asymmetry (Kendall's tau = 0.52, p \textless{} 0.01), consistent with a publication bias in the literature.

To check whether this bias fully explains infants' speech preference, we symmetrized the funnel plot with the ``trim and fill'' method (Duval \& Tweedie, 2000). To symmetrize the funnel plot, 12 (SE = 4.75) missing experiments were needed on the left side of the plot. The corrected effect size was estimated at 0.24 (SE = 0.07) after filling in the 12 missing experiments, which is still significantly different from zero. Thus, even correcting for a potential publication bias, we still find statistical evidence for infants' preferring speech over competitors.

\hypertarget{moderator-analyses}{%
\subsection{Moderator analyses}\label{moderator-analyses}}

We then tested if heterogeneity could be accounted for by the mechanistic explanations described in our introduction. Following our hypotheses, we fit a meta-analytic model with the following moderators:

\begin{itemize}
\tightlist
\item
  mean age of children;
\item
  naturalness of the competitor sound (coded as yes if it was natural and no otherwise);
\item
  vocal quality of the competitor sound (coded as yes if it was vocal and no otherwise);
\item
  familiarity with the language used (native or foreign).
\end{itemize}

These moderators were specified without interactions with each other both to avoid overfitting and because they were sometimes confounded in previous work.

None of the moderators was significant (see Table \ref{tab:Table2}).

\begin{table}[tbp]

\begin{center}
\begin{threeparttable}

\caption{\label{tab:Table2}Statistical results of meta-regression with all moderators. The intercept corresponds to the effect size when the competitor is natural, and vocal, and speech is in a foreign language, at age 0. The moderator estimates correspond to changes in the intercept when the target stimuli are in the native language (familiarity); the competitor is artificial (naturalness); and the competitor is non-vocal (vocal quality).}

\begin{tabular}{lcccc}
\toprule
 & estimate & SE & t & confidence interval\\
\midrule
intercept & 0.17 & 0.17 & 0.98 & -0.19 - 0.54\\
naturalness & 0.27 & 0.13 & 2.02 & -0.02 - 0.55\\
vocal quality & -0.09 & 0.16 & -0.55 & -0.45 - 0.27\\
language & 0.05 & 0.16 & 0.34 & -0.28 - 0.39\\
age & 0.00 & 0.00 & 0.87 & 0 - 0\\
\bottomrule
\end{tabular}

\end{threeparttable}
\end{center}

\end{table}

\begin{figure}
\centering
\includegraphics{MA_speech_pref_files/figure-latex/natural-1.pdf}
\caption{\label{fig:natural}Effect sizes as a function of age and natural quality of the competitor. The size of each dot is inversely proportional to the variance. Positive effect sizes reflect a preference for the speech sound, negative effect sizes reflect a preference for the competitor sound.}
\end{figure}

\begin{table}[tbp]

\begin{center}
\begin{threeparttable}

\caption{\label{tab:TableNatural}Statistical results of meta-regression with naturalness and its interaction with age as moderators. The intercept corresponds to the effect size when the competitor is natural, at age 0. The moderator estimates correspond to changes in the intercept when the competitor is artificial (naturalness).}

\begin{tabular}{lcccc}
\toprule
 & estimate & SE & t & confidence interval\\
\midrule
intercept & 0.36 & 0.21 & 1.67 & -0.17 - 0.88\\
naturalness & 0.00 & 0.24 & 0.00 & -0.53 - 0.52\\
naturalness*age & 0.00 & 0.00 & 1.05 & 0 - 0.01\\
\bottomrule
\end{tabular}

\end{threeparttable}
\end{center}

\end{table}

\begin{figure}
\centering
\includegraphics{MA_speech_pref_files/figure-latex/vocal-1.pdf}
\caption{\label{fig:vocal}Effect sizes as a function of age and vocal quality of the competitor. The size of each dot is inversely proportional to the variance. Positive effect sizes reflect a preference for the speech sound, negative effect sizes reflect a preference for the competitor sound.}
\end{figure}

\begin{table}[tbp]

\begin{center}
\begin{threeparttable}

\caption{\label{tab:TableVocal}Statistical results of meta-regression with vocal quality and its interaction with age as moderators. The intercept corresponds to the effect size when the competitor is vocal, at age 0. The moderator estimates correspond to changes in the intercept when the competitor is non-vocal (vocal quality).}

\begin{tabular}{lcccc}
\toprule
 & estimate & SE & t & confidence interval\\
\midrule
intercept & 0.48 & 0.15 & 3.17 & 0.13 - 0.83\\
vocal quality & -0.40 & 0.22 & -1.86 & -0.86 - 0.06\\
vocal quality*age & 0.00 & 0.00 & 2.40 & 0 - 0.01\\
\bottomrule
\end{tabular}

\end{threeparttable}
\end{center}

\end{table}

\begin{figure}
\centering
\includegraphics{MA_speech_pref_files/figure-latex/lang-1.pdf}
\caption{\label{fig:lang}Effect sizes as a function of age and familiarity with the speech sounds. The size of each dot is inversely proportional to the variance. Positive effect sizes reflect a preference for the speech sound, negative effect sizes reflect a preference for the competitor sound.}
\end{figure}

\begin{table}[tbp]

\begin{center}
\begin{threeparttable}

\caption{\label{tab:TableLang}Statistical results of meta-regression with language of the speech sounds and interaction with age. The intercept corresponds to the effect size when speech is in a foreign language at age 0. The moderator estimate correspond to changes in the intercept when the target stimuli are in the native language.}

\begin{tabular}{lcccc}
\toprule
 & estimate & SE & t & confidence interval\\
\midrule
intercept & 0.72 & 0.30 & 2.44 & 0.03 - 1.41\\
language & -0.48 & 0.32 & -1.52 & -1.19 - 0.22\\
language * age & 0.00 & 0.00 & 2.24 & 0 - 0.01\\
\bottomrule
\end{tabular}

\end{threeparttable}
\end{center}

\end{table}

We also tested each of our three hypotheses by three separate models for each moderator and its interaction with age. There was no significant main effect or interaction (see Figures \ref{fig:natural}, \ref{fig:vocal}, and \ref{fig:lang} ; and Tables \ref{tab:TableNatural}, \ref{tab:TableVocal}, and \ref{tab:TableLang}).

\hypertarget{discussion}{%
\section{Discussion}\label{discussion}}

Our meta-analysis synthesizes the available literature on infants' preference for speech sounds. When all experiments were considered together with no moderators, we found a sizable intercept (g=0.40, g=0.24 when taking the publication bias into account), which was still significant after correcting for the publication bias. Our meta-analysis shows that this preferential processing of speech sounds is observable from birth on. It is important to stress that this conclusion is not trivial because others have said similar things in the past: One key advantage of meta-analysis over conclusions drawn from individual studies or unsystematic narrative reviews is that we can actually measure the likely presence of publication bias (which \emph{is} present in these data, as we discuss below), and furthermore compensate for this bias statistically, to see if an effect remains significant after doing so (\textbf{cristia\_meta-analytic\_2021?}; but see Ioannidis, 2005).

TO DO: add something about which dimensions seem to be confounded in previous work, and the kinds of studies that would tease them apart, as well as how many studies we believe would be necessary to check whether there are interactions (honestly, I think the idea of interactions between naturalness and vocal, or naturalness and familiary do not make sense -- perhaps vocal and familiarity does, but I'm not totally sure)

We had hypothesized age to play a major role, because it may correlate with a reshaping of the category definition for speech itself. Indeed, experiments comparing the preference for human speech against human non-speech as well as animal vocalizations more generally (McDonald et al., 2019; Vouloumanos, Hauser, Werker, \& Martin, 2010) often discuss age-related differences in categorization of these sounds. Surprisingly, age did not significantly moderate the overall preference for speech, as shown by the null estimate of this moderator (Table \ref{tab:Table2}), nor did it interacted with any other moderators (Table \ref{tab:TableLang}, \ref{tab:TableNatural}, and \ref{tab:TableVocal}). This result was replicated in a separate model for age only, which showed a significant intercept, similar to the intercept found in the meta-regression with no moderator (Supplementary results S1). Crucially, age was not centered. This intercept therefore provides an estimate of the effect size at age 0, i.e.~at birth. Moreover, the scatterplot of effect sizes as a function of age reveals clearly no change with age even when plotted without other moderators (Supplementary figure S2). This null effect of age, combined with the sizable intercept, confirms that infants reliably prefer speech over other types of sounds from birth.

The significant heterogeneity we found among the literature suggests that underlying factors modulate this effect. We had predicted infants' speech preference to be larger when the competitor was an artificial sound than when it was a natural one; when the competitor was non-vocal; and when the speech was in the infants' native language. In fact, we were unable to disprove the null hypothesis of no difference for all three factors. Our meta-analysis revealed uneven distributions of experiments across age and stimulus dimensions, and that the distribution of effect sizes in the literature is consistent with publication bias. It is possible that the effect of our three moderators would emerge if the publication biased was solved. However, by aggregating the numerous individual studies of a literature, meta-analyses gain statistical power as compared to individual experiments. As such, meta-analytic results have more cumulative explanatory value than single studies. Moreover, distributions of effect sizes for experiments varying along the three dimensions widely overlap. Our findings therefore suggest that none of these parameters fully explain infants' preference for speech sounds. Our dataset can be community-augmented, and we invite researchers investigating this phenomenon to complement it with any data they would have (\url{https://osf.io/4stz9/?view_only=d0696591ebf34bfc8430f848cd945ca8}), whatever the results and publication status, to solve the publication bias and confirm our findings.

This clearly points to the value of meta-analysis: to take stock of a field and inspire follow-up studies. In particular, future experiments should test infants from 1 to 3 months, and older than 9 months. Language production gains in complexity at about 9 months (Oller, Eilers, Neal, \& Schwartz, 1999), which could affect infants' speech preference. Experiments using natural vocal stimuli as competitor, and foreign speech as target, would contribute to fill in the gap in the litterature that our meta-analysis revealed.

Our results suggest that, from birth on, infants show a preference for speech, which cannot be accounted by the three explanations tested here: naturalness, vocalness, or familiarity with the native language. It is possible that infants prefer speech because of its complex acoustic structure and fast transitions (Rosen \& Iverson, 2007). Spectral or temporal modulations taken separately are not sufficient to elicit neural responses similar to the ones elicited by speech (Minagawa-Kawai, Cristià, Vendelin, Cabrol, \& Dupoux, 2011). However, speech is characterized by joint spectrotemporal modulations at specific rates (Singh \& Theunissen, 2003). It is possible that infants are sensitive to this specific spectro-temporal structure (though see Norman-Haignere \& McDermott, 2018 showing that they only explain neural responses in primary auditory cortex, suggesting that other factors contribute to the behavioral response in later processing stages). Testing this explanation would require to compute the modulation spectra of the actual stimuli used in the experiments. Thus, we recommend interested researchers to deposit their stimuli in a public archive such as the Open Science Framework (Foster \& Deardorff, 2017).

A capacity to preferentially listen to speech sounds from birth is compatible with the idea that infants are born with the capacity to recognize their conspecifics' communication signals. This parallels what has been proposed for faces: Infants are born with the capacity to orient their attention toward them, even without any prior exposure to faces (Morton \& Johnson, 1991; Turati, 2004).
\textbf{ac2ci to address R2's comment 8, I changed ``suggests that'' with ``is compatible with the idea that''}
This would stem from basic perceptual abilities present at birth, namely that the visual system would be tuned to a spatial structure that correspond to those of faces (Morton \& Johnson, 1991; Turati, 2004). As newborns have never been exposed to such visual stimuli before, it would reflect general properties (i.e.~filters) of the visual system. Similarly, the auditory system could be tuned to a spectro-temporal structure that speech presents. The combination of this non-specific bias with the systematic variations of the auditory environment (i.e.~the fact that an acoustical structure characterizes speech but not other sounds of the environment) would result in preferential responses to speech from birth. However, contrary to faces, fetuses are exposed to speech as low-pass filtered by the womb throughout the last trimester of gestation (Lecanuet \& Granier-Deferre, 1993; Querleu, Renard, Versyp, Paris-Delrue, \& Crèpin, 1988). It is therefore possible that prenatal experience with low-pass filtered speech helps infants to form a representation of speech, by tuning the response properties of the auditory system to speech.
The fact that familiarity with the language used in the experiment did not modulate infants' preference suggests that this effect is not triggered by familiarity with the sounds of the native language. Infants would therefore form a representation that is specific enough to discriminate speech from other natural or vocal sounds, but general enough to be independent of the language spoken.

The human species is a gregarious one. Detecting speech signals allows to integrate it with other sensory percepts, such as faces, to form multisensory representations of conspecifics. Five-months old infants were capable to match human faces to speech, as well as monkey faces to monkey vocalizations (Vouloumanos, Druhen, Hauser, \& Huizink, 2009). This provides evidence that human infants make correspondences between faces and vocalizations, and that they distinguish their conspecifics from other species. This lays the groundwork for social cognition. Finally, the preference itself may also be a meaningful index of processing that can be used to identify children at risk (Sorcinelli, Ference, Curtin, \& Vouloumanos, 2019). Understanding this phenomenon is therefore crucial for both theoretical and clinical advances.

Ultimately, preferential processing of speech may support higher level cognitive tasks. Identifying speech signals and paying attention to them would allow infants to form complex representations of the sensory world, that they can manipulate cognitively. Consistently, infants could categorize visual stimuli (i.e., associate a label to a category of objects) when they were associated to speech, but not pure tones or backward speech (Ferry, Hespos, \& Waxman, 2010, 2013; Fulkerson \& Waxman, 2007). Interestingly, infants categorized visual stimuli when presented with speech, melodies, monkey (Fulkerson \& Haaf, 2003), or lemur vocalizations (Ferry, Hespos, \& Waxman, 2013). These results support the idea that infants' cognition is set up to respond to, and manipulate complex sounds (i.e.~modulated in time and frequency), especially those of critical ecological importance such as communicative vocalizations.

With a sample size of 791 infants, covering the wide age range of all individual experiments available, our meta-analysis provides evidence for a specialized processing of speech from birth. This parallels infants' attention to faces from birth (Morton \& Johnson, 1991; Turati, 2004), and suggests that human cognition is set up to pay attention to signals from conspecifics specifically. Such systems may be the precursors of humans' advanced social cognition skills.

\newpage

\hypertarget{references}{%
\section{References}\label{references}}

\begingroup
\setlength{\parindent}{-0.5in}
\setlength{\leftskip}{0.5in}

References marked with an asterisk indicate studies included in the meta-analysis.

\hypertarget{refs}{}
\begin{CSLReferences}{1}{0}
\leavevmode\hypertarget{ref-bergmann_development_2016}{}%
Bergmann, C., \& Cristia, A. (2016). Development of infants' segmentation of words from native speech: A meta-analytic approach. \emph{Developmental Science}, \emph{19}(6), 901--917. \url{https://doi.org/10.1111/desc.12341}

\leavevmode\hypertarget{ref-bergmann_promoting_2018}{}%
Bergmann, C., Tsuji, S., Piccinini, P. E., Lewis, M. L., Braginsky, M., Frank, M. C., \& Cristia, A. (2018). Promoting {Replicability} in {Developmental} {Research} {Through} {Meta}-analyses: {Insights} {From} {Language} {Acquisition} {Research}. \emph{Child Development}, \emph{89}(6), 1996--2009. \url{https://doi.org/10.1111/cdev.13079}

\leavevmode\hypertarget{ref-black_quantifying_2017}{}%
Black, A., \& Bergmann, C. (2017). Quantifying infants' statistical word segmentation: {A} meta-analysis (pp. 124--129). Cognitive Science Society. Retrieved from \url{https://pure.mpg.de/pubman/faces/ViewItemOverviewPage.jsp?itemId=item_2475527}

\leavevmode\hypertarget{ref-borenstein_introduction_2011}{}%
Borenstein, M., Hedges, L. V., Higgins, J. P. T., \& Rothstein, H. R. (2011). \emph{Introduction to {Meta}-{Analysis}}. John Wiley \& Sons.

\leavevmode\hypertarget{ref-colombo_method_1981}{}%
Colombo, J., \& Bundy, R. S. (1981). A method for the measurement of infant auditory selectivity. \emph{Infant Behavior and Development}, \emph{4}, 219--223. \url{https://doi.org/10.1016/S0163-6383(81)80025-2}

\leavevmode\hypertarget{ref-cooper_developmental_1994}{}%
Cooper, R. P., \& Aslin, R. N. (1994). Developmental {Differences} in {Infant} {Attention} to the {Spectral} {Properties} of {Infant}-{Directed} {Speech}. \emph{Child Development}, \emph{65}(6), 1663--1677. \url{https://doi.org/10.2307/1131286}

\leavevmode\hypertarget{ref-curtin_speech_2013}{}%
Curtin, S., \& Vouloumanos, A. (2013). Speech {Preference} is {Associated} with {Autistic}-{Like} {Behavior} in 18-{Months}-{Olds} at {Risk} for {Autism} {Spectrum} {Disorder}. \emph{Journal of Autism and Developmental Disorders}, \emph{43}(9), 2114--2120. \url{https://doi.org/10.1007/s10803-013-1759-1}

\leavevmode\hypertarget{ref-dunlap_meta-analysis_1996}{}%
Dunlap, W. P., Cortina, J. M., Vaslow, J. B., \& Burke, M. J. (1996). Meta-analysis of experiments with matched groups or repeated measures designs. \emph{Psychological Methods}, \emph{1}(2), 170--177. \url{https://doi.org/10.1037/1082-989X.1.2.170}

\leavevmode\hypertarget{ref-duval_trim_2000}{}%
Duval, S., \& Tweedie, R. (2000). Trim and {Fill}: {A} {Simple} {Funnel}-{Plot}--{Based} {Method} of {Testing} and {Adjusting} for {Publication} {Bias} in {Meta}-{Analysis}. \emph{Biometrics}, \emph{56}(2), 455--463. \url{https://doi.org/10.1111/j.0006-341X.2000.00455.x}

\leavevmode\hypertarget{ref-ecklund-flores_asymmetric_1996}{}%
Ecklund-Flores, L., \& Turkewitz, G. (1996). Asymmetric headturning to speech and nonspeech in human newborns. \emph{Developmental Psychobiology}, \emph{29}(3), 205--217. \url{https://doi.org/10.1002/(SICI)1098-2302(199604)29:3\%3C205::AID-DEV2\%3E3.0.CO;2-V}

\leavevmode\hypertarget{ref-ference_role_2018}{}%
Ference, J. D. (2018). The {Role} of {Attentional} {Biases} for {Conspecific} {Vocalizations}. https://doi.org/\url{http://dx.doi.org/10.11575/PRISM/31878}

\leavevmode\hypertarget{ref-ferry_categorization_2010}{}%
Ferry, A. L., Hespos, S. J., \& Waxman, S. R. (2010). Categorization in 3- and 4-{Month}-{Old} {Infants}: {An} {Advantage} of {Words} {Over} {Tones}. \emph{Child Development}, \emph{81}(2), 472--479. \url{https://doi.org/10.1111/j.1467-8624.2009.01408.x}

\leavevmode\hypertarget{ref-ferry_nonhuman_2013}{}%
Ferry, A. L., Hespos, S. J., \& Waxman, S. R. (2013). Nonhuman primate vocalizations support categorization in very young human infants. \emph{Proceedings of the National Academy of Sciences of the United States of America}, \emph{110}(38), 15231--15235. \url{https://doi.org/10.1073/pnas.1221166110}

\leavevmode\hypertarget{ref-foster_open_2017}{}%
Foster, E. D., \& Deardorff, A. (2017). Open {Science} {Framework} ({OSF}). \emph{Journal of the Medical Library Association : JMLA}, \emph{105}(2), 203--206. \url{https://doi.org/10.5195/jmla.2017.88}

\leavevmode\hypertarget{ref-fulkerson_influence_2003}{}%
Fulkerson, A. L., \& Haaf, R. A. (2003). The {Influence} of {Labels}, {Non}-{Labeling} {Sounds}, and {Source} of {Auditory} {Input} on 9- and 15-{Month}-{Olds}' {Object} {Categorization}. \emph{Infancy}, \emph{4}(3), 349--369. \url{https://doi.org/10.1207/S15327078IN0403_03}

\leavevmode\hypertarget{ref-fulkerson_words_2007}{}%
Fulkerson, A. L., \& Waxman, S. R. (2007). Words (but not {Tones}) facilitate object categorization: {Evidence} from 6- and 12-month-olds. \emph{Cognition}, \emph{105}(1), 218--228. \url{https://doi.org/10.1016/j.cognition.2006.09.005}

\leavevmode\hypertarget{ref-gelman_difference_2006}{}%
Gelman, A., \& Stern, H. (2006). The {Difference} {Between} {``{Significant}''} and {``{Not} {Significant}''} is not {Itself} {Statistically} {Significant}. \emph{The American Statistician}, \emph{60}(4), 328--331. \url{https://doi.org/10.1198/000313006X152649}

\leavevmode\hypertarget{ref-hedges_robust_2010}{}%
Hedges, L. V., Tipton, E., \& Johnson, M. C. (2010). Robust variance estimation in meta-regression with dependent effect size estimates. \emph{Research Synthesis Methods}, \emph{1}(1), 39--65. \url{https://doi.org/10.1002/jrsm.5}

\leavevmode\hypertarget{ref-hunter_multifactor_1988}{}%
Hunter, M. A., \& Ames, E. W. (1988). A multifactor model of infant preferences for novel and familiar stimuli. \emph{Advances in Infancy Research}, \emph{5}, 69--95.

\leavevmode\hypertarget{ref-ioannidis_why_2005}{}%
Ioannidis, J. P. A. (2005). Why {Most} {Published} {Research} {Findings} {Are} {False}. \emph{PLOS Medicine}, \emph{2}(8), e124. \url{https://doi.org/10.1371/journal.pmed.0020124}

\leavevmode\hypertarget{ref-lecanuet_speech_1993}{}%
Lecanuet, J.-P., \& Granier-Deferre, C. (1993). Speech {Stimuli} in the {Fetal} {Environment}. In B. de Boysson-Bardies, S. de Schonen, P. Jusczyk, P. McNeilage, \& J. Morton (Eds.), \emph{Developmental {Neurocognition}: {Speech} and {Face} {Processing} in the {First} {Year} of {Life}} (pp. 237--248). Dordrecht: Springer Netherlands. \url{https://doi.org/10.1007/978-94-015-8234-6_20}

\leavevmode\hypertarget{ref-lipsey_practical_2001}{}%
Lipsey, M. W., \& Wilson, D. B. (2001). \emph{Practical meta-analysis}. Thousand Oaks, CA, US: Sage Publications, Inc.

\leavevmode\hypertarget{ref-mcdonald_infant_2019}{}%
McDonald, N. M., Perdue, K. L., Eilbott, J., Loyal, J., Shic, F., \& Pelphrey, K. A. (2019). Infant brain responses to social sounds: {A} longitudinal functional near-infrared spectroscopy study. \emph{Developmental Cognitive Neuroscience}, \emph{36}, 100638. \url{https://doi.org/10.1016/j.dcn.2019.100638}

\leavevmode\hypertarget{ref-mehler_precursor_1988}{}%
Mehler, J., Jusczyk, P., Lambertz, G., Halsted, N., Bertoncini, J., \& Amiel-Tison, C. (1988). A precursor of language acquisition in young infants. \emph{Cognition}, \emph{29}(2), 143--178. \url{https://doi.org/10.1016/0010-0277(88)90035-2}

\leavevmode\hypertarget{ref-minagawa-kawai_assessing_2011}{}%
Minagawa-Kawai, Y., Cristià, A., Vendelin, I., Cabrol, D., \& Dupoux, E. (2011). Assessing {Signal}-{Driven} {Mechanisms} in {Neonates}: {Brain} {Responses} to {Temporally} and {Spectrally} {Different} {Sounds}. \emph{Frontiers in Psychology}, \emph{2}. \url{https://doi.org/10.3389/fpsyg.2011.00135}

\leavevmode\hypertarget{ref-mizrahi_single_2014}{}%
Mizrahi, A., Shalev, A., \& Nelken, I. (2014). Single neuron and population coding of natural sounds in auditory cortex. \emph{Current Opinion in Neurobiology}, \emph{24}, 103--110. \url{https://doi.org/10.1016/j.conb.2013.09.007}

\leavevmode\hypertarget{ref-moher_preferred_2009}{}%
Moher, D., Liberati, A., Tetzlaff, J., Altman, D. G., \& Group, T. P. (2009). Preferred {Reporting} {Items} for {Systematic} {Reviews} and {Meta}-{Analyses}: {The} {PRISMA} {Statement}. \emph{PLOS Medicine}, \emph{6}(7), e1000097. \url{https://doi.org/10.1371/journal.pmed.1000097}

\leavevmode\hypertarget{ref-moon_two-day-olds_1993}{}%
Moon, C., Cooper, R. P., \& Fifer, W. P. (1993). Two-day-olds prefer their native language. \emph{Infant Behavior and Development}, \emph{16}(4), 495--500. \url{https://doi.org/10.1016/0163-6383(93)80007-U}

\leavevmode\hypertarget{ref-morton_conspec_1991}{}%
Morton, J., \& Johnson, M. H. (1991). {CONSPEC} and {CONLERN}: {A} {Two}-{Process} {Theory} of {Infant} {Face} {Recognition}. \emph{Psychological Review}, \emph{98}(2), 164--181.

\leavevmode\hypertarget{ref-norman-haignere_neural_2018}{}%
Norman-Haignere, S. V., \& McDermott, J. H. (2018). Neural responses to natural and model-matched stimuli reveal distinct computations in primary and nonprimary auditory cortex. \emph{PLOS Biology}, \emph{16}(12), e2005127. \url{https://doi.org/10.1371/journal.pbio.2005127}

\leavevmode\hypertarget{ref-oakes_sample_2017}{}%
Oakes, L. M. (2017). Sample {Size}, {Statistical} {Power}, and {False} {Conclusions} in {Infant} {Looking}-{Time} {Research}. \emph{Infancy}, \emph{22}(4), 436--469. \url{https://doi.org/10.1111/infa.12186}

\leavevmode\hypertarget{ref-oller_precursors_1999}{}%
Oller, D. K., Eilers, R. E., Neal, A. R., \& Schwartz, H. K. (1999). Precursors to speech in infancy: {The} prediction of speech and language disorders. \emph{Journal of Communication Disorders}, \emph{32}(4), 223--245. \url{https://doi.org/10.1016/S0021-9924(99)00013-1}

\leavevmode\hypertarget{ref-querleu_fetal_1988}{}%
Querleu, D., Renard, X., Versyp, F., Paris-Delrue, L., \& Crèpin, G. (1988). Fetal hearing. \emph{European Journal of Obstetrics \& Gynecology and Reproductive Biology}, \emph{28}(3), 191--212. \url{https://doi.org/10.1016/0028-2243(88)90030-5}

\leavevmode\hypertarget{ref-r_core_team_r:_2018}{}%
R Core Team. (2018). R: {A} {Language} and {Environment} for {Statistical} {Computing}. Vienna, Austria: R Foundation for Statistical Computing. Retrieved from \url{https://www.R-project.org}

\leavevmode\hypertarget{ref-rosen_constructing_2007}{}%
Rosen, S., \& Iverson, P. (2007). Constructing adequate non-speech analogues: What is special about speech anyway? \emph{Developmental Science}, \emph{10}(2), 165--168. \url{https://doi.org/10.1111/j.1467-7687.2007.00550.x}

\leavevmode\hypertarget{ref-santolin_role_2019}{}%
Santolin, C., Russo, S., Calignano, G., Saffran, J. R., \& Valenza, E. (2019). The role of prosody in infants' preference for speech: {A} comparison between speech and birdsong. \emph{Infancy}, \emph{24}(5), 827--833. \url{https://doi.org/10.1111/infa.12295}

\leavevmode\hypertarget{ref-santolin_infants_2020}{}%
Santolin, C., Zettersten, M., \& Saffran, J. R. (2020). Infants' preference for non-native speech versus birdsong. {Unpublished} data. Retrieved from \url{https://github.com/mzettersten/birdsong}

\leavevmode\hypertarget{ref-segal_listening_2011}{}%
Segal, O., \& Kishon-Rabin, L. (2011). Listening {Preference} for {Child}-{Directed} {Speech} {Versus} {Nonspeech} {Stimuli} in {Normal}-{Hearing} and {Hearing}-{Impaired} {Infants} {After} {Cochlear} {Implantation} {{}} {Ovid}. \emph{Ear and Hearing}, \emph{32}(3), 358--372. \url{https://doi.org/10.1097/AUD.0b013e3182008afc}

\leavevmode\hypertarget{ref-segal_infants_2021}{}%
Segal, O., Kligler, N., \& Kishon-Rabin, L. (2021). Infants' {Preference} for {Child}-{Directed} {Speech} {Over} {Time}-{Reversed} {Speech} in {On}-{Channel} and {Off}-{Channel} {Masking}. \emph{Journal of Speech, Language \& Hearing Research}, \emph{64}(7), 2897--2908. \url{https://doi.org/10.1044/2021_JSLHR-20-00279}

\leavevmode\hypertarget{ref-shultz_three-month-olds_2010}{}%
Shultz, S., \& Vouloumanos, A. (2010). Three-{Month}-{Olds} {Prefer} {Speech} to {Other} {Naturally} {Occurring} {Signals}. \emph{Language Learning and Development}, \emph{6}(4), 241--257. \url{https://doi.org/10.1080/15475440903507830}

\leavevmode\hypertarget{ref-singh_modulation_2003}{}%
Singh, N. C., \& Theunissen, F. E. (2003). Modulation spectra of natural sounds and ethological theories of auditory processing. \emph{The Journal of the Acoustical Society of America}, \emph{114}(6), 3394. \url{https://doi.org/10.1121/1.1624067}

\leavevmode\hypertarget{ref-smith_efficient_2006}{}%
Smith, E. C., \& Lewicki, M. S. (2006). Efficient auditory coding. \emph{Nature}, \emph{439}(7079), 978--982. \url{https://doi.org/10.1038/nature04485}

\leavevmode\hypertarget{ref-sorcinelli_preference_2019}{}%
Sorcinelli, A., Ference, J., Curtin, S., \& Vouloumanos, A. (2019). Preference for speech in infancy differentially predicts language skills and autism-like behaviors. \emph{Journal of Experimental Child Psychology}, \emph{178}, 295--316. \url{https://doi.org/10.1016/j.jecp.2018.09.011}

\leavevmode\hypertarget{ref-spence_prenatal_1987}{}%
Spence, M. J., \& DeCasper, A. J. (1987). Prenatal experience with low-frequency maternal-voice sounds influence neonatal perception of maternal voice samples. \emph{Infant Behavior and Development}, \emph{10}(2), 133--142. \url{https://doi.org/10.1016/0163-6383(87)90028-2}

\leavevmode\hypertarget{ref-turati_why_2004}{}%
Turati, C. (2004). Why {Faces} {Are} {Not} {Special} to {Newborns}: {An} {Alternative} {Account} of the {Face} {Preference}. \emph{Current Directions in Psychological Science}, \emph{13}(1), 5--8. \url{https://doi.org/10.1111/j.0963-7214.2004.01301002.x}

\leavevmode\hypertarget{ref-vanden_bosch_der_nederlanden_infant_2021}{}%
Vanden Bosch der Nederlanden, C. M., \& Vouloumanos, A. (2021). Infant biases for detecting speech in complex scenes. \emph{Developmental Psychology}, \emph{57}(9), 1411--1422. \url{https://doi.org/10.1037/dev0000974}

\leavevmode\hypertarget{ref-vouloumanos_foundational_2014}{}%
Vouloumanos, A., \& Curtin, S. (2014). Foundational {Tuning}: {How} {Infants}' {Attention} to {Speech} {Predicts} {Language} {Development}. \emph{Cognitive Science}, \emph{38}(8), 1675--1686. \url{https://doi.org/10.1111/cogs.12128}

\leavevmode\hypertarget{ref-vouloumanos_five-month-old_2009}{}%
Vouloumanos, A., Druhen, M. J., Hauser, M. D., \& Huizink, A. T. (2009). Five-month-old infants' identification of the sources of vocalizations. \emph{Proceedings of the National Academy of Sciences of the United States of America}, \emph{106}(44), 18867--18872. \url{https://doi.org/10.1073/pnas.0906049106}

\leavevmode\hypertarget{ref-vouloumanos_tuning_2010}{}%
Vouloumanos, A., Hauser, M. D., Werker, J. F., \& Martin, A. (2010). The tuning of human neonates' preference for speech. \emph{Child Development}, \emph{81}(2), 517--527. \url{https://doi.org/10.1111/j.1467-8624.2009.01412.x}

\leavevmode\hypertarget{ref-vouloumanos_listen_2014}{}%
Vouloumanos, A., \& Waxman, S. R. (2014). Listen up! {Speech} is for thinking during infancy. \emph{Trends in Cognitive Sciences}, \emph{18}(12), 642--646. \url{https://doi.org/10.1016/j.tics.2014.10.001}

\leavevmode\hypertarget{ref-vouloumanos_tuned_2004}{}%
Vouloumanos, A., \& Werker, J. F. (2004). Tuned to the signal: The privileged status of speech for young infants. \emph{Developmental Science}, \emph{7}(3), 270--276. \url{https://doi.org/10.1111/j.1467-7687.2004.00345.x}

\leavevmode\hypertarget{ref-vouloumanos_listening_2007}{}%
Vouloumanos, A., \& Werker, J. F. (2007). Listening to language at birth: Evidence for a bias for speech in neonates. \emph{Developmental Science}, \emph{10}(2), 159--164. \url{https://doi.org/10.1111/j.1467-7687.2007.00549.x}

\leavevmode\hypertarget{ref-yamashiro_does_2020}{}%
Yamashiro, A., Curtin, S., \& Vouloumanos, A. (2020). Does an {Early} {Speech} {Preference} {Predict} {Linguistic} and {Social}-{Pragmatic} {Attention} in {Infants} {Displaying} and {Not} {Displaying} {Later} {ASD} {Symptoms}? \emph{Journal of Autism and Developmental Disorders}, \emph{50}(7), 2475--2490. \url{https://doi.org/10.1007/s10803-019-03924-2}

\end{CSLReferences}

\endgroup


\end{document}
